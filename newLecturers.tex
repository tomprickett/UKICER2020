%%
%% This is file `sample-sigplan.tex',
%% generated with the docstrip utility.
%%
%% The original source files were:
%%
%% samples.dtx  (with options: `sigplan')
%% 
%% IMPORTANT NOTICE:
%% 
%% For the copyright see the source file.
%% 
%% Any modified versions of this file must be renamed
%% with new filenames distinct from sample-sigplan.tex.
%% 
%% For distribution of the original source see the terms
%% for copying and modification in the file samples.dtx.
%% 
%% This generated file may be distributed as long as the
%% original source files, as listed above, are part of the
%% same distribution. (The sources need not necessarily be
%% in the same archive or directory.)
%%
%% The first command in your LaTeX source must be the \documentclass command.
\documentclass[sigplan,screen]{acmart}
%\settopmatter{authorsperrow=2}
% NOTE that a single column version is required for submission and peer review. This can be done by changing the \doucmentclass[...]{acmart} in this template 
%2
 %to 
 
%\documentclass[manuscript,screen]{acmart}

%%
%% \BibTeX command to typeset BibTeX logo in the docs
\AtBeginDocument{%
  \providecommand\BibTeX{{%
    \normalfont B\kern-0.5em{\scshape i\kern-0.25em b}\kern-0.8em\TeX}}}

%% Rights management information.  This information is sent to you
%% when you complete the rights form.  These commands have SAMPLE
%% values in them; it is your responsibility as an author to replace
%% the commands and values with those provided to you when you
%% complete the rights form.
\setcopyright{acmcopyright}
\copyrightyear{2020}
\acmYear{2020}
%\acmDOI{10.1145/1122445.1122456}

%% These commands are for a PROCEEDINGS abstract or paper.
\acmConference[UKICER '20]{UKICER20: United Kingdom and Ireland Computing Education Research Conference}{September 03--04, 2020}{Glasgow, UK}
\acmBooktitle{UKICER '20: United Kingdom and Ireland Computing
  Education Research Conference,
  September 03--04, 2020, Glasgow, UK}
%\acmPrice{15.00}
%\acmISBN{978-1-4503-XXXX-X/XX/XX}


%%
%% Submission ID.
%% Use this when submitting an article to a sponsored event. You'll
%% receive a unique submission ID from the organizers
%% of the event, and this ID should be used as the parameter to this command.
%%\acmSubmissionID{123-A56-BU3}

%%
%% The majority of ACM publications use numbered citations and
%% references.  The command \citestyle{authoryear} switches to the
%% "author year" style.
%%
%% If you are preparing content for an event
%% sponsored by ACM SIGGRAPH, you must use the "author year" style of
%% citations and references.
%% Uncommenting
%% the next command will enable that style.
%%\citestyle{acmauthoryear}

%%
%% end of the preamble, start of the body of the document source.
\begin{document}

%%
%% The "title" command has an optional parameter,
%% allowing the author to define a "short title" to be used in page headers.
\title{Integrating New Research Faculty into the\\UK Computer Science Education Community}

%%
%% The "author" command and its associated commands are used to define
%% the authors and their affiliations.
%% Of note is the shared affiliation of the first two authors, and the
%% "authornote" and "authornotemark" commands
%% used to denote shared contribution to the research.



 \author{Alan Hayes}
 \author{James H. Davenport}
 \affiliation{%
   \institution{ University of Bath}
   \city{Bath}
   \country{UK}
 }
 \email{a.hayes@bath.ac.uk}
 \email{j.h.davenport@bath.ac.uk}
 
 \author{Tom Crick}
 \affiliation{
   \institution{Swansea University}
   \city{Swansea}
   \country{UK}
 }
 \email{thomas.crick@swansea.ac.uk}

 \author{Alastair Irons}
 \affiliation{
   \institution{ Sunderland University}
   \city{Sunderland}
   \country{UK} }
 \email{alastair.irons@sunderland.ac.uk}

 \author{Tom Prickett}
 \affiliation{
   \institution{ Northumbria University}
   \city{Newcastle upon Tyne}
   \country{UK}
 }
 \email{tom.prickett@northumbria.ac.uk}


%%
%% By default, the full list of authors will be used in the page
%% headers. Often, this list is too long, and will overlap
%% other information printed in the page headers. This command allows
%% the author to define a more concise list
%% of authors' names for this purpose.
%\renewcommand{\shortauthors}{Crick, et al.}

%%
%% The abstract is a short summary of the work to be presented in the
%% article.
\begin{abstract}
A vibrant Computer Science Education (CSE) Community of Practice is
emerging in the United Kingdom and Ireland (UK\&I), promoted by national
and international professional body/learned society specialist
interest groups and
supported through a number of CSE research and
practice conferences.  This workshop explores how this community of practice addresses the needs of new academics to
UK\&I higher education and what opportunities there
are to bring together and develop new academics as part of this
community.
  
The complex and contested demands of learning and teaching in UK\&I
higher education make the early career of an academic challenging ~\cite{Thomas2015} and potentially lonely ~\cite{Foote2009},
especially when balanced against their research aspirations, and wider professional service
commitments. Learning and teaching
development in the UK commonly involves working
towards Fellowship of the Higher Education Academy~\cite{fellowship}
(now known as AdvanceHE), either by direct application or by an
accredited university postgraduate course. Typically, this is
supported by mentoring from within a department. The quality of
learning provided will be promoted in part by the
strength of the community of practice operating within the
department~\cite{Bolander2008} and the communities of practice that
exist at a national and international level~\cite{Thomas2015}.

At the workshop attendees will participate in a qualitative
research exercise, to shape and develop
a proposal to further promote the value of new academics engaging with CSE research to enhance their own learning and teaching activities. In particular, addressing:

\begin{enumerate}
% \item What is the background and previous career experience of new CS
%   lecturers?
\item What current and future opportunities are there to engage new CS faculty
  with the UK\&I CSE community of practice?
% \item What further opportunities are there to welcome new CS lectures
% to the national CSE community of practice
\item What is the potential to supplement institutional
academic/research development opportunities with national
developmental opportunities?
\item How can we continue to raise the profile and value of new
CS research faculty engaging with CSE research?
\end{enumerate}

The workshop would consists of two hours, online remote activity.
By attending, early-career computer science faculty would gain a better appreciation for the
opportunities there are to engage and obtain support from the
UK\&I CSE research community. Experienced academics, would gain a better appreciation for the development needs of their colleagues and how this augments their research and learning
and teaching roles.  All participants will have the opportunity to help shape a proposal for enhanced support for new academics.

\end{abstract}

%%
%% The code below is generated by the tool at http://dl.acm.org/ccs.cfm.
%% Please copy and paste the code instead of the example below.
%%
\begin{CCSXML}
<ccs2012>
<concept>
<concept_id>10003456.10003457.10003527</concept_id>
<concept_desc>Social and professional topics~Computing education</concept_desc>
<concept_significance>500</concept_significance>
</concept>
</ccs2012>
\end{CCSXML}

\ccsdesc[500]{Social and professional topics~Computing education}
%%
%% Keywords. The author(s) should pick words that accurately describe
%% the work being presented. Separate the keywords with commas.
\keywords{computer science education, community, induction}

%% A "teaser" image appears between the author and affiliation
%% information and the body of the document, and typically spans the
%% page.
% \begin{comment}

% \begin{teaserfigure}
%   \includegraphics[width=\textwidth]{sampleteaser}
%   \caption{Seattle Mariners at Spring Training, 2010.}
%   \Description{Enjoying the baseball game from the third-base
%   seats. Ichiro Suzuki preparing to bat.}
%   \label{fig:teaser}
% \end{teaserfigure}
% \end{comment}

%%
%% This command processes the author and affiliation and title
%% information and builds the first part of the formatted document.
\maketitle






%%
%% The acknowledgments section is defined using the "acks" environment
%% (and NOT an unnumbered section). This ensures the proper
%% identification of the section in the article metadata, and the
%% consistent spelling of the heading.
% \begin{acks}
% There is the potential for the Institute of Coding (IoC) to provide support for the proposal that emerges from this workshop.
% \end{acks}
\begin{acks}
All the authors' institutions are members of the Institute of Coding (IoC), an initiative funded by the Office for Students (England) and the Higher Education Funding Council for Wales. The BCS Academy of Computing and the IoC has supported the development of this workshop and the IoC will consider providing support for the proposal that emerges from this workshop.
\end{acks}
    

%%
%% The next two lines define the bibliography style to be used, and
%% the bibliography file.
\bibliographystyle{ACM-Reference-Format}
\bibliography{newLecturerCourse}

%%
%% If your work has an appendix, this is the place to put it.
\begin{comment}


\appendix

\end{comment}

\end{document}
\endinput
%%
%% End of file `sample-sigplan.tex'.
